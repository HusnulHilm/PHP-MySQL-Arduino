\section{Einleitung}
%

Im Abschnitt Motivation  wird definiert, warum es Sinn macht, Ger"ate z.b Kompressor 
im Industrie 4.0 zu vernetzen.Es wird zuerst die technische L"osungen gezeigt. 
Im Abschnitt Zielsetzung wird erkl"art, was man mit dieser Arbeit 
erreicht wird und welche Bestandteil daf"ur benutzt wird.
\subsection{Zusammenfassung}

Industrie 4.0 ist das Schlagwort,welches die 4.Industrielle Revolution beschreibt.
Ger"ate und Maschinen werden heute Intelligent durch die Vernetzung miteinander und
 mit den Menschen "uber das Internet kommunizieren.
Maschinen und Ger"ate werden in der Zukunft digitalisiert 
\footnote{Digitalisierung bedeutet in dieser Arbeit die Vernetzung der einzelnen Komponenten untereinander.} 
und mit Sensoren aufgebaut sein.
Sie kommunizieren immer mit verschiedenen Systemen.
 Zum Beispiel mit
Entwicklung, Produktion  sowie Lieferanten und Kunden.
Damit das alles funktioniert, m"ussen Sensoren in Maschinen
digital erfasst werden. Es gibt einen neuen Begriff,
der das Internet of Things (IoT) hei"st, um die Sensoren
und andere Maschinen mit digitalen Informationen zu verbinden.
Die Bewertung und Simulationen werden "uber das Internet funktionieren.
Dank der Vernetzung in der Industrie 4.0 k"onnten Menschen viele
genaue Informationen von verschiedenen Quellen erhalten und falls
es St"orungen gibt k"onnten diese schnell bearbeitet werden und gleichzeitig 
die richtigen Entscheidungen getroffen werden. 
Diese Entwicklung verbessert viele Prozesse in unterschiedlichen
Bereichen und vereinfacht die Kommunikation mit anderen Abteilungen.[1]
\subsection{Zielsetzung}

Das Ziel dieser Arbeit ist die Konzeption und Entwicklung einer
mechatronischen Betriebsumgebung f"ur einen Kompressoren-Versuchsstand
in Anlehnung an Industrie 4.0.
Daf"ur soll ein mechatroniches Konzept und Auswertunggsoftware 
entwickelt werden.Das Auslesen der Sensoren erfolgt "uber einen Arduino,
mit der Plattform eigenen Programmiersprache.Die Auswertung der Daten 
soll dann mit Hilfe von PHPausgewertetund mit MYSQL gespeichert werden.Die
Auswertung soll dem Anwender anschlie"send grapisch zur Verf"ugung stehen.
 

Das Layout folgt in den Kapiteln Projektplanung, Software Engineering
und UI-Prototypen. Weil Software Engineering grundlegend f"ur die
Software-Entwicklung ist und die Nutzbarkeit Hand in Hand mit der
Software-Entwicklung funktioniert, deswegen ist es wichtig,
sich eingehend mit dem Entwicklungs-Modell/Prozess zu befassen.

