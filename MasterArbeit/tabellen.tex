\section{Tabellen}

Und hier wollen wir ein paar Tabellen erzeugen:

\subsection{Einfache Tabellen}

\begin{tabular}{cc}
 x & y \\
 -4 & 16 \\
 -3 & 9 \\
 -2 & 4 \\
 -1 & 1 \\
  0 & 0 \\
  1 & 1 \\
  2 & 4 \\
  3 & 9 \\
  4 & 16 \\
\end{tabular}


  und hier ein bisschen anders

\begin{tabular}{|c|r|}
\hline
 x & y \\
\hline
 -4 & 16 \\
 -3 & 9 \\
 -2 & 4 \\
 -1 & 1 \\
  0 & 0 \\
  1 & 1 \\
  2 & 4 \\
  3 & 9 \\
  4 & 16 \\
\hline
\end{tabular}

\subsection{Gleitende Tabellen}

Eine Tabelle kann auch als Gleitobjekt ausgef"uhrt werden:

\begin{table}[htbp]
\begin{center}
\begin{tabular}{|cc|}
\hline
 x & y \\
\hline
 -4 & 16 \\
 -3 & 9 \\
 -2 & 4 \\
 -1 & 1 \\
  0 & 0 \\
  1 & 1 \\
  2 & 4 \\
  3 & 9 \\
  4 & 16 \\
\hline
\end{tabular}
\end{center}
\caption{Wertetabelle f"ur die Normalparabel}\label{tab:einfach1}
\end{table}



Auf die Tabelle kann nun direkt im Text zugegriffen werden, indem man schreibt, dass Tabelle~\ref{tab:einfach1} die Anfangsdaten zum Zeichnen einer Normalparabel enth"alt. Dabei steht diese Tabelle auf der Seite~\pageref{tab:einfach1}.
