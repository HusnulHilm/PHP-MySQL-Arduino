\section{Entwicklung der mechatronischen Systeme}

Der Japaner Ko Kilucki, 
Pr"asident von YASKAWA, 
hat im Jahr 1969 das Wort Mechatronik erfunden, 
was f"ur Hersteller die Erweiterung der 
elektronischen Funktion der mechanischen Komponente bedeutet, 
die aus Mechanik und Elektronik besteht, ein Begriff, 
der dem Betreiber von 1971 bis 1982 gegeben wurde. 
Mit dem Entstehen der Mikroelektronik und der Mikroprozessortechnik 
ist die Informationstechnologie heute ein weiterer 
Element der Mechatronik. 
Die Mechatronik setzt auf die Synergien von Maschinenbau, 
Elektrotechnik und Informationstechnik. 
Die mechatronischen Systeme repr"asentieren die 
Funktionalit"at von Sensoren, Aktoren und Signalverarbeitung. 
Es ist m"oglich, dass das Basissystem mechanische, 
chemische und elektrische Komponenten enth"alt. 
Ziel der Mechatronik ist es, das Verhalten technischer Systeme 
durch Sensoren zu verbessern, 
die Signale liefern und die schrittweise verarbeiten. 
Die Erg"anzung von Produkten mit moderner Informationstechnologie 
kann zu flexiblen technischen Systemen f"uhren. 
Diese Systeme k"onnen auf Ver"anderungen in ihrer Umgebung 
reagieren und kritische Prozesse entdecken, 
die durch den Einsatz von Regelungstechnik optimiert werden[23].

