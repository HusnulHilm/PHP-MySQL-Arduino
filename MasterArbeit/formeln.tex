\section{Formelsatz}

Hier schauen wir uns den Formelsatz an.
Cafe im $\frac{3}{4}$-Takt.

\subsection{Nummerierte Gleichungen}

In Latex wird eine Gleichung automatisch nummeriert:
\begin{equation}\label{eq:paran}
 f(x) = x^2
\end{equation}
Wie zu sehen, stellt die Gleichung~\ref{eq:paran} eine Normalparabel dar.
Die k"onnen wir auch integrieren:
\begin{equation}\label{eq:intpara}
 F(x) = \int x^2 dx = \frac{x^3}{3} + C
\end{equation}
Und auch diese weitere Gleichung~\ref{eq:intpara} kann referenziert werden.

\subsection{Gleichungen ohne Nummerierung}

Gleichungen ohne nummerierung werden durch eine andere Umgebung dargestellt:

\begin{displaymath}
 c = \sqrt{a^2 + b^2}
\end{displaymath}
In dieser Umgebung kann nicht referenziert werden.
